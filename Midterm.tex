\documentclass{article}
\usepackage{amsmath}
\usepackage{amssymb}
\usepackage{outlines}
\usepackage[margin=1in]{geometry}

\begin{document}

\title{Midterm Review}
\author{Kongsak Tipakornrojanakit [5680624]}
\date{}
\maketitle


\textbf{Formulas:}


\begin{alignat}{3}
	\label{EPA}		&\mbox{Equal Payment Amortization} 		&\mbox{C} 			&=  \frac{\mbox{PVA}}{\mbox{PVIFA}(r, t)}\\
	\label{FVIF}	&\mbox{Future Value Interest Factor} 	&\mbox{FVIF(r, t)} 	&=  (1 + r)^t\\ 
	\label{PVIF}	&\mbox{Present Value Interest Factor} 	&\mbox{PVIF(r, t)} 	&=  \frac{1}{(1 + r)^t}
%	&\mbox{Equal Payment }  &\mbox{C} =  \frac{\mbox{PVA}}{\mbox{PVIFA}(r, t)}
\end{alignat}







\begin{outline}[enumerate]
\1 
	\2 
		$ C_{50-months} = \dfrac{\mbox{PVA}}{\mbox{PVIFA}(r, t)} = \dfrac{2,139,423}{\mbox{PVIFA}\left(1.01, 62 - 12 \right)} $
		$ 	= \dfrac{2,139,423}{39.196} = 54,582.5233 \mbox{ Baht per Month}$ \\\\
		$\mbox{Total is } 54,582.5233 \times 50 = 2,729,126.167 \mbox{ Baht}$ \\\\
		$ C_{62-months} = \dfrac{2,729,126.167}{62} = 44,018.16398 \mbox{ Baht} $
	\2 	$ \mbox{PVA}_{24-months} = c \times {\mbox{PVIFA}(r, t)} = 50,000 \times {\mbox{PVIFA}(1, 24)} $ \\\\
		$ \mbox{PVA}_{24-months} = 1,062,169.363 $ \\\\
		$ \mbox{PVA}_{24-months-onward} = 2,139,425 - 1,062,169.363 = 1,077,255.637 $ \\\\
		$ 1,077,255.637 = 50,000 \times {\mbox{PVIFA}(2, 24)} $ \\\\
		$ 1,077,255.637 = 50,000 \times \dfrac{\left[1 - \dfrac{1}{(1 + 0.02)^t}\right]}{0.02} $ \\\\
		$ 1,077,255.637 = 2,500,000 \times \left[1 - \dfrac{1}{(1.02)^t}\right] $ \\\\
		$ 0.4309022548 = 1 - \dfrac{1}{(1.02)^t} $ \\\\
		$ \dfrac{1}{0.5690977452} = (1.02)^t $ \\\\
		$ \log_{1.02}\left(\dfrac{1}{0.5690977452}\right) = \log_{1.02}(1.02)^t $ \\\\
		$ 28.46607508 = t $ \\\\
		$ \mbox{Total Months} = 24 + 28.46607508 = 52.46607508 \mbox{ months} $ \\\\

\newpage
\1
	\2 	$\mbox{FVIF}_{\tiny\mbox{BBL}} = \mbox{FVIF}\left(\dfrac{8.95}{6}, 1 \times 6\right) = 1.0929$ \\\\
		$\mbox{FVIF}_{\tiny\mbox{SCB}} = \mbox{FVIF}\left(\dfrac{9.00}{3}, 1 \times 3\right) = 1.0927$ \\\\
		$\mbox{FVIF}_{\tiny\mbox{TFB}} = \mbox{FVIF}\left(\dfrac{9.05}{2}, 1 \times 2\right) = 1.0925$ \\\\
		Since $\mbox{FVIF}_{\tiny\mbox{TFB}}$ has the lowest \textbf{Effective Annual Rate}; therefore, it is the cheapest choice to go with.\\

	\2 \textbf{Note: Only the first two years }\\
		$ C = \dfrac{\mbox{PVA}}{\mbox{PVIFA}(r, t)} = \dfrac{1.6 \times 10^6}{\mbox{PVIFA}\left(\dfrac{9.00}{3}, 10 \times 3 \right)} $
		$ 	= \dfrac{1.6 \times 10^6}{19.60} =  81,630.8149 \mbox{ Baht per 4 Months}$\\\\
		$\mbox{Total for the first two years } = 81,630.8149 \times 3 \times 2 = 489,784.8895 \mbox{ Baht} $
		$\mbox{Let } F = 489,784.8895$ \\\\
        $ \because I_0 + I_1 + \dots + I_n = (B_0 + B_1 + \dots + B_n) \times r$ \\\\
        $ \because B_0 + B_1 + \dots + B_n = \dfrac{(B_0 + B_n) \times n}{2} $ \\\\
        $ \therefore I_0 + I_1 + \dots + I_n = \dfrac{(B_0 + B_n) \times n \times r}{2} $ \\\\
        $ \because B_n = B_0 - F $ \\\\
        $ \therefore \sum_{x = 0}^{n} I_x =\dfrac{(2B_0 - F) \times n \times r}{2} $ 


\begin{align*}
         \sum_{x = 0}^{6} I_x &= \dfrac{(2(1.6 \times 10^6) - 489,784.8895) \times 6 \times \dfrac{0.09}{3}}{2} \\
        &= ((3.2 \times 10^6) - 489,784.8895) \times 0.09 \\
        &= 243,919.36 \\\\
        \sum_{x = 0}^{6} P_x &= F - \sum_{x = 0}^{6} I_x \\
        &= 489,784.8895 - 243,919.36 \\
        &= 245,865.5296
\end{align*}


		
		
		
\end{outline}





\end{document}